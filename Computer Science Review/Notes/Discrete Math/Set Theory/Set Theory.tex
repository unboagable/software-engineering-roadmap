% You should title the file with a .tex extension (hw1.tex, for example)
\documentclass[11pt]{article}

\usepackage{amsmath}
\usepackage{amssymb}
\usepackage{fancyhdr}
\usepackage{listings}
\usepackage{color}
\usepackage{graphicx}
\graphicspath{ {images/} }
\usepackage{hyperref}
\usepackage{mathtools}

\definecolor{dkgreen}{rgb}{0,0.6,0}
\definecolor{gray}{rgb}{0.5,0.5,0.5}
\definecolor{mauve}{rgb}{0.58,0,0.82}

\lstset{frame=tb,
  language=Java,
  aboveskip=3mm,
  belowskip=3mm,
  showstringspaces=false,
  columns=flexible,
  basicstyle={\small\ttfamily},
  numbers=none,
  numberstyle=\tiny\color{gray},
  keywordstyle=\color{blue},
  commentstyle=\color{dkgreen},
  stringstyle=\color{mauve},
  breaklines=true,
  breakatwhitespace=true,
  tabsize=3
}

\oddsidemargin0cm
\topmargin-2cm     %I recommend adding these three lines to increase the 
\textwidth16.5cm   %amount of usable space on the page (and save trees)
\textheight23.5cm  

\newcommand{\question}[2] {\vspace{.25in} \hrule\vspace{0.5em}
\noindent{\bf #1: #2} \vspace{0.5em}
\hrule \vspace{.10in}}
\renewcommand{\part}[1] {\vspace{.10in} {\bf (#1)}}

\newcommand{\myname}{Chang-Hyun Mungai}
\newcommand{\myhwnum}{Set Theory Notes}

\setlength{\parindent}{0pt}
\setlength{\parskip}{5pt plus 1pt}
 
\pagestyle{fancyplain}
\lhead{\fancyplain{}{\textbf{\myhwnum}}}      % Note the different brackets!

\begin{document}

\medskip                        % Skip a "medium" amount of space
                                % (latex determines what medium is)
                                % Also try: \bigskip, \littleskip

\thispagestyle{plain}
\begin{center}                  % Center the following lines
{\Large Set Theory} \\
\end{center}

\question{Definitions}

\begin{itemize}
  \item an object either belongs or does not belong
  \item set-collection of things that are brought together because they obey a certain (well defined) rule
\begin{itemize}
  \item ex: numbers, people, shapes
\end{itemize}
  \item element- "thing" that belongs to a given set
\end{itemize}

\question{Symbols}

\begin{itemize}
  \item $\{$...$\}$-the set of ...
\begin{itemize}
  \item ex:$\{$-3,-2,-1,0,1,2$\}$, $\{$integers between -3 and 3 inclusive$\}$, $\{$x$\mid$x is an integer amd $\mid$x$\mid$ \textless 4$\}$
\end{itemize}
  \item ($\in$) symbol means element of
  \item set usually uppercase A,B and elements lowercase x,y
  \item U-universal set, all things under discussion
  \item $\{\}$,($\varnothing$)-empty set, null set
  \item N-natural numbers, whole numbers starting at 1
  \item Z-integers
  \item R-real numbers
\end{itemize}

\question{Set Operations}

\begin{itemize}
  \item (A $\cap$ B)intersection (two sets overlap)
  \item (A $\cup$ B) union (elements in either)
  \item (A - B) or (A \ B) difference (elements in A but not B)
  \item (A') or($A^C$) or  complement (everything not in A is in A')
  \item cardinality (if A=$\{$lowercase letters of the alphabet$\}$, $\mid$ A$\mid$ =26)
  \item P(A) powerset-set of all subsets (including empty) of A
\begin{itemize}
  \item if $\mid$ A $\mid$=k then $\mid$ P(A )$\mid$ =$2^k$ proof (for each element we can choose to include element or not)
\end{itemize}
  \item Cartesian Products
\begin{itemize}
  \item if we have n sets: A1 , A2, ..., An, then their Cartesian product is defined by: $A1 × A2 × ... × An$ = $\{$ (a1, a2, ..., an) $\mid$ a1 $\in$ A1, a2 $\in$ A2, ..., an $\in$ An) $\}$ and (a1, a2, ..., an) is called an ordered n-tuple.
\end{itemize}
\end{itemize}

\question{Relationships}

\begin{itemize}
  \item Equality = (same elements, repeats ignored)
  \item subsets (A$\subseteq$B) all elements of A are also elements of B
\begin{itemize}
  \item A $\subseteq$ B and B $\subseteq$ A, then A = B
  \item proper subset (A $\subset$ B) if B contains at least one element that isn't in A
\end{itemize}
  \item disjoint no elements in common
\end{itemize}

\question{Foundational Rules of Set Theory}

\begin{itemize}
  \item The Laws of Sets
  \begin{itemize}
    \item Commutative Laws
    \begin{itemize}
       \item $\cap$ B = B $\cap$ A
       \item A $\cup$ B = B $\cup$ A
    \end{itemize}
    \item Associative Laws
    \begin{itemize}
       \item (A $\cap$ B) $\cap$ C = A $\cap$ (B $\cap$ C)
       \item (A $\cup$ B) $\cup$ C = A $\cup$ (B $\cup$ C)
    \end{itemize}
    \item Distributive Laws
    \begin{itemize}
       \item A $\cap$ (B $\cup$ C) = (A $\cap$ B) $\cup$ (A $\cap$ C)
       \item A $\cup$ (B $\cap$ C) = (A $\cup$ B) $\cap$ (A $\cup$ C)
    \end{itemize}
    \item Idempotent Laws
    \begin{itemize}
       \item A $\cap$ A = A
       \item A $\cup$ A = A
    \end{itemize}
    \item Identity Laws
    \begin{itemize}
       \item A $\cup$ $\varnothing$ = A
       \item A $\cap$ U = A
       \item A $\cup$ U = U
       \item A $\cap$ $\varnothing$ = $\varnothing$
    \end{itemize}
    \item Involution Law
    \begin{itemize}
       \item (A ') ' = A
    \end{itemize}
    \item Complement Laws
    \begin{itemize}
       \item A $\cup$ A' = U
       \item A $\cap$ A' = $\varnothing$
       \item U ' = $\varnothing$
       \item $\varnothing$ ' = U
    \end{itemize}
    \item De Morgan’s Laws
    \begin{itemize}
       \item (A $\cap$ B) ' = A ' $\cup$ B '
       \item (A $\cup$ B) ' = A ' $\cap$ B '
    \item proof
    \begin{itemize}
       \item (A $\cup$ B) ' $\subseteq$ A ' $\cap$ B '
       \item A' $\cap$ B ' $\subseteq$ (A $\cup$ B) '
    \end{itemize}
    \end{itemize}
  \end{itemize}
\end{itemize}

\question{sources}

\begin{itemize}
\item \url{https://en.wikibooks.org/wiki/Discrete_Mathematics/Set_theory}
\end{itemize}

\end{document}

