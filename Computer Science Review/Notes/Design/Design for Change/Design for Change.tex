% You should title the file with a .tex extension (hw1.tex, for example)
\documentclass[11pt]{article}

\usepackage{amsmath}
\usepackage{amssymb}
\usepackage{fancyhdr}
\usepackage{listings}
\usepackage{color}
\usepackage{graphicx}
\graphicspath{ {images/} }
\usepackage{hyperref}
\usepackage{mathtools}

\definecolor{dkgreen}{rgb}{0,0.6,0}
\definecolor{gray}{rgb}{0.5,0.5,0.5}
\definecolor{mauve}{rgb}{0.58,0,0.82}

\lstset{frame=tb,
  language=Java,
  aboveskip=3mm,
  belowskip=3mm,
  showstringspaces=false,
  columns=flexible,
  basicstyle={\small\ttfamily},
  numbers=none,
  numberstyle=\tiny\color{gray},
  keywordstyle=\color{blue},
  commentstyle=\color{dkgreen},
  stringstyle=\color{mauve},
  breaklines=true,
  breakatwhitespace=true,
  tabsize=3
}

\oddsidemargin0cm
\topmargin-2cm     %I recommend adding these three lines to increase the 
\textwidth16.5cm   %amount of usable space on the page (and save trees)
\textheight23.5cm  

\newcommand{\question}[2] {\vspace{.25in} \hrule\vspace{0.5em}
\noindent{\bf #1: #2} \vspace{0.5em}
\hrule \vspace{.10in}}
\renewcommand{\part}[1] {\vspace{.10in} {\bf (#1)}}

\newcommand{\myname}{Chang-Hyun Mungai}
\newcommand{\myandrew}{cmungai@andrew.cmu.edu}
\newcommand{\myhwnum}{Design for Change notes}

\setlength{\parindent}{0pt}
\setlength{\parskip}{5pt plus 1pt}
 
\pagestyle{fancyplain}
\lhead{\fancyplain{}{\textbf{\myhwnum}}}      % Note the different brackets!

\begin{document}

\medskip                        % Skip a "medium" amount of space
                                % (latex determines what medium is)
                                % Also try: \bigskip, \littleskip

\thispagestyle{plain}
\begin{center}                  % Center the following lines
{\Large Design for Change} \\
\end{center}

\question{Design principle for change: information hiding}

\begin{itemize}
\item expose little implementation as possible
\item allows you to change hidden details later
\end{itemize}

\question{Subtype Polymorphism}

\begin{itemize}
  \item There may be multiple implementations of an interface
  \item Multiple implementations coexist in the same program
  \item May not even be distinguishable
  \item Every object has its own data and behavior
\end{itemize}

\question{Interface}

\begin{itemize}
  \item can implement start (defining required methods and classes)
  \item create class to implement interface
\end{itemize}

\question{What to test}

\begin{itemize}
  \item Functional correctness of a method (e.g.,computations, contracts)
  \item Functional correctness of a class (e.g., class invariants)
  \item Behavior of a class in a subsystem/multiple subsystems/the entire system
  \item Behavior when interacting with the world
\begin{itemize}
  \item Interacting with files, networks, sensors, …
  \item Erroneous states
  \item Nondeterminism, Parallelism
  \item Interaction with users
\end{itemize}
  \item Other qualities (performance, robustness, usability, security, …)
\end{itemize}

\question{Unit Tests (Junit good for JAVA)}

\begin{itemize}
  \item Unit tests for small units: functions, classes, subsystems
\begin{itemize}
  \item Smallest testable part of a system
  \item Test parts before assembling them
  \item Intended to catch local bugs
\end{itemize}
  \item Typically written by developers
  \item Many small, fast‐running, independent tests
  \item Little dependencies on other system parts or environment
  \item Insufficient but a good starting point
  \item extra benefits:
\begin{itemize}
  \item Documentation (executable specification)
  \item Design mechanism (design for testability)
\end{itemize}
\end{itemize}

\question{Test cases strategies}

\begin{itemize}
  \item use specs
  \item representative cases
  \item invalid cases
  \item boundary cond
  \item think like attacker
  \item dificult cases
\end{itemize}

\question{static methods}

\begin{itemize}
  \item Static methods belong to a class
  \item global
  \item Direct dispatch, no subtype polymorphism
  \item Avoid unless really only a single implementation exists (e.g., Math.min)
\end{itemize}

\question{Best practices}

\begin{itemize}
  \item control access
\begin{itemize}
  \item fields not accessible from client code
  \item methods only accessible in exposed interface
\end{itemize}
  \item contracts - agreement between provider and user
\begin{itemize}
  \item interface specification
  \item functionality and correctness expectations
  \item Performance expectations
\end{itemize}
  \item Visibility Modifiers
\begin{itemize}
  \item design principle for change: information hiding
  \item expose little implementation as possible
  \item allows you to change hidden details later
\end{itemize}
\end{itemize}

\question{Notes}

\begin{itemize}
  \item try to avoid setters
  \item Organize program functionality around kinds of abstract “objects”
\begin{itemize}
  \item For each object kind, offer a specific set of operations on the objects
  \item Objects are otherwise opaque: Details of representation are hidden
  \item “Messages to the receiving object”
\end{itemize}
  \item Distinguish interface from class
\begin{itemize}
  \item Interface: expectations
  \item Class: delivery on expectations (the implementation)
  \item Anonymous class: special Java construct to create objects without explicit classes: Point x = new Point() { /* implementation */ };
\end{itemize}
  \item Explicitly represent the taxonomy of object types
\begin{itemize}
  \item This is the type hierarchy (!= inheritance, more on that later): A CartesianPoint is a Point
\end{itemize}
  \item Design Patterns!!
\begin{itemize}
  \item Design Patterns by Gamma,Helm,Johnson,Vlissides
\end{itemize}
\end{itemize}

\question{sources}

\begin{itemize}
\item \url{http://www.cs.cmu.edu/~charlie/courses/15-214/2015-fall/index.html#schedule}
\end{itemize}

\end{document}

