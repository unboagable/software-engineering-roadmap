% You should title the file with a .tex extension (hw1.tex, for example)
\documentclass[11pt]{article}

\usepackage{amsmath}
\usepackage{amssymb}
\usepackage{fancyhdr}
\usepackage{listings}
\usepackage{color}
\usepackage{graphicx}
\graphicspath{ {images/} }
\usepackage{hyperref}
\usepackage{mathtools}

\definecolor{dkgreen}{rgb}{0,0.6,0}
\definecolor{gray}{rgb}{0.5,0.5,0.5}
\definecolor{mauve}{rgb}{0.58,0,0.82}

\lstset{frame=tb,
  language=Java,
  aboveskip=3mm,
  belowskip=3mm,
  showstringspaces=false,
  columns=flexible,
  basicstyle={\small\ttfamily},
  numbers=none,
  numberstyle=\tiny\color{gray},
  keywordstyle=\color{blue},
  commentstyle=\color{dkgreen},
  stringstyle=\color{mauve},
  breaklines=true,
  breakatwhitespace=true,
  tabsize=3
}

\oddsidemargin0cm
\topmargin-2cm     %I recommend adding these three lines to increase the 
\textwidth16.5cm   %amount of usable space on the page (and save trees)
\textheight23.5cm  

\newcommand{\question}[2] {\vspace{.25in} \hrule\vspace{0.5em}
\noindent{\bf #1: #2} \vspace{0.5em}
\hrule \vspace{.10in}}
\renewcommand{\part}[1] {\vspace{.10in} {\bf (#1)}}

\newcommand{\myname}{Chang-Hyun Mungai}
\newcommand{\myhwnum}{Nuances Notes}

\setlength{\parindent}{0pt}
\setlength{\parskip}{5pt plus 1pt}
 
\pagestyle{fancyplain}
\lhead{\fancyplain{}{\textbf{\myhwnum}}}      % Note the different brackets!

\begin{document}

\medskip                        % Skip a "medium" amount of space
                                % (latex determines what medium is)
                                % Also try: \bigskip, \littleskip

\thispagestyle{plain}
\begin{center}                  % Center the following lines
{\Large Nuances} \\
\end{center}

\question{General Tips}

\begin{itemize}
  \item Getter and setter  
  \item Override and super
  \item Java automatically collects garbage  
  \item \&\&/$\mid\mid$ checks left first
  \item + strings makes a new string every time, if you want to do in a loop use stringbuilder(reduce memory)
  \item Everything in Java not explicitly set to something, is initialized to a zero value
  \begin{itemize}
    \item references (anything that holds an object):null
    \item int/short/byte:0
    \item float/double:0.0
    \item booleans: false.
    \item array of something, all entries are also zeroed
  \end{itemize}
\end{itemize}

\question{Virtual}

\begin{itemize}
  \item a virtual function (or method) is a function whose behavior can be overridden within an inheriting class by a function with the same signature to provide the polymorphic behavior
  \item according to definition, every non-static method in JAVA is by default virtual method except final and private methods
\end{itemize}

\question{Switch Statement}

\begin{itemize}
  \item All matching cases will be run unless their is a break statement
  \item Example
  \begin{itemize}
  \item 
    \begin{lstlisting}
switch (month) {
            case 1:  monthString = "January";
                     break;
            case 2:  monthString = "February";
                     break;
            case 3:  monthString = "March";
                     break;
            case 4:  monthString = "April";
                     break;
            case 5:  monthString = "May";
                     break;
            case 6:  monthString = "June";
                     break;
            case 7:  monthString = "July";
                     break;
            case 8:  monthString = "August";
                     break;
            case 9:  monthString = "September";
                     break;
            case 10: monthString = "October";
                     break;
            case 11: monthString = "November";
                     break;
            case 12: monthString = "December";
                     break;
            default: monthString = "Invalid month";
                     break;
        }
    \end{lstlisting}
  \end{itemize}
\end{itemize}

\question{Breaking out of for loops}

\begin{itemize}
  \item if you want to skip a particular iteration, use continue
  \begin{itemize}
    \item

\begin{lstlisting}
for(int i=0 ; i<5 ; i++){

    if (i==2){

      continue;
    }
}
\end{lstlisting}

\end{itemize}
  \item if you want to break out of the immediate loop use break
  \begin{itemize}
    \item

\begin{lstlisting}
for(int i=0 ; i<5 ; i++){

    if (i==2){

        break;
    }
}
\end{lstlisting}

  \end{itemize}
  \item if there are 2 loop, outer and inner.... and you want to break out of both the loop from  the inner loop, use break with label
  \begin{itemize}
    \item

\begin{lstlisting}
lab1: for(int j=0 ; j<5 ; j++){
     for(int i=0 ; i<5 ; i++){

        if (i==2){

          break lab1;
        }
     }
}
\end{lstlisting}

  \end{itemize}
\end{itemize}

\question{Things to override in new object (for hashing and equality uses)}

  \begin{itemize}
    \item public int hashCode()
    \item public boolean equals(Object object)
  \end{itemize}

\begin{lstlisting}
ex: Tiger
	@Override
	public boolean equals(Object object) {
		boolean result = false;
		if (object == null || object.getClass() != getClass()) {
			result = false;
		} else {
			Tiger tiger = (Tiger) object;
			if (this.color == tiger.getColor()
					&& this.stripePattern == tiger.getStripePattern()) {
				result = true;
			}
		}
		return result;
	}
\end{lstlisting}

\question{Useful built in functions}

\begin{itemize}
  \item Arrays
\begin{itemize}
  \item Arrays.binarySearch(arr, target)
  \begin{itemize}
    \item Negative value shows where it should be
  \end{itemize}
  \item Arrays.sort(arr)
\end{itemize}
\end{itemize}

\question{Sources}

\begin{itemize}
\item \url{https://www.cs.utexas.edu/~scottm/cs307/codingSamples.htm}
\item \url{https://www.jitendrazaa.com/blog/java/virtual-function-in-java/}
\end{itemize}

\end{document}

